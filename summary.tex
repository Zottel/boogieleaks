\documentclass[paper=a4, fontsize=12pt]{scrartcl}

\title{	
  \rule{\linewidth}{1px} \\[0.4cm]
  \textsc{Computer-Aided Verification} \\ [25pt]
  \huge Project summary \\
  \rule{\linewidth}{1px} \\[0.4cm]
  \textsc{Boehm Simon} \\ [12pt]
  \textsc{Danger Daniel} \\ [12pt]
  \textsc{Roob Julius} \\ [12pt]
  \rule{\linewidth}{1px} \\[0.4cm]
}

\begin{document}
\maketitle 

\section{Project goal}
In a short sentences the goal of our project is the re-implementation of the Boogie verification condition generator. 

Boogie is a Intermediate Verification Language published by Microsoft Research which is designed to make the prescription of verification conditions natural and convenient. The architecture of Boogie is build to verify Spec\# programs in the object oriented .Net framework. This is done by a pipeline performing a series of transformations from the source program in Spec\# over Boogie and verification conditions to an error report.

Our goal is the part of this pipeline where the verification conditions are generated from the Intermediate Verification Language Boogie. This can be divided into two sub-parts: the parsing of the Boogie program code and the generation of verification conditions. The verification conditions should then be solved by a SMT solver, e.g. Z3.

We decided to use Python as programming language for this project, because it overs a lot of great and easy to use libraries. 

\section{Current status}
There are mainly two Boogie papers. "This is Boogie 2" and "Boogie: A Modular Reusable Verifier for Object-Oriented Programs". For our work "This is Boogie 2" is the main source because it contains the whole grammar and semantics of the Boogie language."Boogie: A Modular Reusable Verifier for Object-Oriented Programs" describes more the techniques used to convert Spec\# sources to Boogie programs, which is not part of our project. Another source is the source code of the Boogie implementation from Microsoft Research, which is available under the Microsoft Public License (Ms-PL). But because there are a lot of additional features the code is hard to understand and not easy to reuse for our purposes.

At the moment we work on the parsing of the Boogie source code. For this we use Ply which is the Python implementation of the compiler construction tools lex and yacc. Lex is used to generate a lexical analyser, a so called lexer, which divides the Boogie source code into single tokens and yacc is used to generate the parser. The implementation of all the Boogie grammar rule in the notation of yacc, which is similar to BNF, costs a lot of time and was very error-prone so we decide to do this work all together as group to control each other.

\section{Timeline}
After the parsing of the Boogie code is working the next step will be the construction of an abstract syntax tree (AST). And with the help of the AST we could finally create the verification conditions. This verification conditions should use the SMT-LIB so that different SMT solvers can be used to evaluate them. 

\end{document}

